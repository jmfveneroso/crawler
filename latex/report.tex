\documentclass{acmart}
\usepackage{graphicx}
\usepackage[utf8]{inputenc}
\usepackage[T1]{fontenc}

\begin{document}

\title{Information Retrieval Assignment 1 Report}
\author{João Mateus de Freitas Veneroso}

\maketitle

\begin{abstract}
This report makes a thorough examination of a web crawler engine implemented
in C/C++.
\end{abstract}

\section{Introduction}

A web crawler is a computer software designed to retrieve documents by following
hyperlinks just as a internet user would do. In order to achieve 
good coverage the crawler needs to make efficient usage of the available bandwidth.
This report describes a crawler programmed in C/C++ which aimed to retrieve a broad
sample of the brazilian network and its latest release can be found at: 
https://github.com/jmfveneroso/crawler. For installation instructions and usage 
examples see the README.md file. 

\section{Project}
The aim of this project was to build an efficient command line tool to collect a 
significant portion of the brazilian web. The end result achieved a performance of
up to 300 page downloads per second in ideal conditions and averaged about 190 pages 
per second. Note that this performance was quite possibly capped by the available bandwidth.
The crawler's operation can be summarised into four steps:

\begin{description}
  \item [Read]     The next url is read from the queue.
  \item [Fetch]    A web page is retrieved. Multiple fetchers can perform this task in parallel.
  \item [Schedule] All valid links found in a document are scheduled to be crawled next.
  \item [Write]    The document is written to an output file.
\end{description}

\subsection{Read}

During the read phase the next url to be crawled is retrieved according to its priority, which 
is set during the "schedule" phase. However multiple requests to the same server (url root) in rapid 
succession are avoided in order to prevent making the server unavailable to its actual users.
This is achieved by following a simple politeness policy: if any request has been made to the
requested server in the last 30 seconds the url is added to a delayed queue and the next one
is fetched until a target that abides to the politeness policy is found.

The read operation is atomic.

\subsection{Fetch}

Latency varies broadly among web servers. This means that once a request has been made it may 
take several seconds before the server responds, and during that time gap the processor becomes 
idle resulting in loss of performance. To achieve better results the fetch operation can be 
performed by multiple threads operating in parallel.
After the fetching operation terminates, multiple threads must compete to schedule the newly 
found hyperlinks and persist the retrieved document.

\subsection{Schedule}

The scheduling phase register urls found in hyperlinks and sorts them according to their priority.
Since our main goal is to retrieve a broad sample of documents written in portuguese, urls without 
the suffix ".br" were discarded from our collection. A language checker would be a better 
alternative in order to achieve this goal, but is falls out of the scope of this project due 
to time limitation.

Even when considering brazilian web pages alone the sample size becomes huge, so there is a need to 
impose a priority check before performing the retrieval. In order to so, unspidered urls
were classified in categories according to word sizes ranging from 1 to 10. Any sequence of 
characters occurring between "/" was considered a complete word, for example:

\begin{itemize}
\item "www.example.com.br" has size 1
\item "www.example.com.br/category" has size 2
\item "www.example.uol.com.br" has size 1
\end{itemize}

If the url size exceeded 10 words, it was classified in the tenth category. It was observed that
the overwhelming majority of urls had 9 words or less, so this decision did not have a huge impact 
in the crawler's operation.

We evaluated the possibility of checking word size before the first ocurrence of "/" but it seemed
arbitrary to give greater priority to "www.example.com.br" compared to "www.example.br". The consequence
is that "www.app.example.com" is assigned the same priority as "www.example.com" in our crawler.

This operation is atomic, so multiple threads cannot register urls at the same time.

\subsection{Write}

The main goal of this project is to build a sample of the brazilian network so all downloaded documents must be 
persisted to secondary memory. This operation is a simple write to disk operation and it is atomic. The finishing
file size can reach several gigabytes after a few minutes running.

\section{Experiments}

Three experiments were conducted to evaluate trade offs between architecture choices and 
to find the characteristics of the collection of documents available in portuguese over the web. The
experiments are described below.

\subsection{Optimal thread performance}
Multipled threads were used during the fetch phase but the optimal number of concurrent processes was
not evident. The goal of this experiment was to find the number of threads in which the ratio
of pages downloaded per second was maximized.
The number of threads performing the fetch operation was varied from 8 to 512 in increments of size 8. And at
each iteration 1000 urls were downloaded. The remaining crawler operations were kept unvaried through the 
experiment.
The sample size may be small, but it is somewhat irrelevant for the end sought. The sample size has impact
over the atomic operations of the crawler, not the fetch phase. So if a number of threads is optimal for
this small sample size it can be argued that it will remain so for larger collections.

The total time and the collection size were recorded for each number of threads providing the following results:

[Graph goes here]

By assessing the experimental data it was possible to notice that there exists a trade off between document size 
and speed of retrieval. A big number of threads performs considerably faster but the collection size diminishes 
as the number of threads increases.
This happens due to the server response time. When many threads are fetching web pages from multiple servers,
the smaller documents will have a faster response time, so the collection will end up with a smaller average size.
When the number of threads is small, most threads will get stuck waiting for a big document or waiting for the
server to timeout and as a consequence the collection size will decrease.

Since our goal in this project is to collect as many pages as possible in as little a time as possible
we chose to run all other experiments with 512 parallel threads.

\subsection{Primary and secondary memory storage}
In order to obtain huge sample sizes it is virtually impossible to keep all data structures residing in 
primary memory so the goal of this experiment was to find out if there was a huge performance loss when 
secondary memory was used for this goal.
The experiment consisted in crawling 100.000 urls while keeping all relevant data structures first in
primary and then in secondary memory.

The project implements two data structures that may reach considerable sizes: the url hash table and the 
url priority list.
  
\subsubsection{Url hash table}
The url hash table ascertains url uniqueness when the crawler is evaluating which link to crawl next. This
structure is of uttermost importance to avoid downloading the same document over and over again. The table
functions also as a key value store in which the values correspond to the last time that url was queried, a
necessary feature to make the politeness policy achievable.

In secondary memory, the url hash table was modeled by making use of open addressing and linear probing as
a method of conflict resolution. 
This means that in case of collision, the key is stored in the next available space. This design must
store entries in fixed size buckets, so all urls were considered to have at most 256 characters. This is 
inefficient considering that most urls are much smaller than that but it works well as a proof of concept. The retrieval
time for this model is very efficient when the table is less than half filled but decreases rapidly after that.
To make sure retrieval times were fast enough the table size was made to be much larger than the amount of urls stored.

In primary memory the url hash table was implemented with the stl container std::set.

\subsubsection{Url priority list}
The priority list was called that way to differentiate it from the stl container priority queue. 
It actually consists of multiple lists, one for each priority level, with constant insertion complexity.

In secondary memory the priority list was modeled as 10 separate files in which the urls were written one after another
without any underlying structure. The file cursor was recorded to keep track where the next sequence of reads had to be
performed and 10000 urls were kept in primary memory at each time. The new urls are simply written to the end of the file.

In primary memory the url priority list was modeled as 10 instances of the stl container std::queue.

We found that eventhough primary memory is faster, the performance loss of using secondary memory was not
very significant. The bandwidth seems to present a more significant barrier when ultra fast networks are
not available so the next experiment was conducted by storing the data structures in secondary memory.

\subsubsection{Massive Crawling}
In order to acquire a big collection size, the crawler was left running for 24 hours. 1.231.444 web pages were
downloaded and the collection size reached around 10 gigabytes.

[data goes here]

The average size of a page in the collection is 60kbs. The average number of crawlable links found in each page 
is 100 which means that at each step the amount of crawlable targets increases tenfold. However this rate of increase
would probably face some reduction as the collection size increases.

[pages/s by time graph]

The crawler experienced little oscillation in the average speed of retrieval as the time passed. 
It averaged at about 200 pages per second, amounting to a total bandwidth of 96 megabits per second, 
roughly a quarter of the available bandwidth if it were dedicated entirely to this task.
Also, part of the total available bandwidth has to be employed necessarily in downloading the 
robots.txt file and receiving failed responses (non 200 http codes) so even in perfect conditions
it is not possible to make use of all the available connection speed.

The hash table size for this experiment was 4gb. At the end of the experiment, the hash table was 80% full.
The url priority list held 100000 urls yet to be crawled and they were distributed in the following order
of priority:

[graph with urls and order of priority]

This shows that our assumption that most urls would fit in the first 9 priority levels was right. In total
only 0.5% of all urls were composed of 10 or more words.

\section{Conclusion}
The experiments showed our crawler achieved a good performance peak and kept stable velocities all through the
program's execution, even when the data structures were written to secondary memory. As a proof of concept,
this project showed the guidelines described in this report are sound enough to model a more robust web
crawler.

\end{document}
